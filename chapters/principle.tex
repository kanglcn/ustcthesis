\chapter{InSAR技术原理}

\section{SAR简介}

合成孔径雷达(Synthetic apeture radar,SAR)
是一种可以产生高分辨率图像的卫星雷达系统,属于一种微波成像雷达。
SAR由传统的雷达技术发展而来。
通过距离向(range direction)和方位向(azimuth direction)的压缩成像技术能够实现以较小的雷达天线
得到较高空间分辨率的影像。
针对不同的数据使用目的,
一般SAR传感器有三种成像模式,
区别在于接收天线的几何位置关系以及数据获取方式。
\begin{enumerate}
    \item 单轨双天线横向模式。
    所谓单轨,意思是只做一次观测,但需要有两副天线搭载在飞行器上,同时接受两份数据。
    横向的意思是两幅天线和飞行轨道垂直。
    如图\ref{fig:sar1}所示。
    单轨双天线的好处在于两副天线之间的距离可以精确确定,
    常用于地形的测量。
    \begin{figure}[htp]
        \centering
        \includegraphics[width=0.8\textwidth]{sar1.png}
        \caption{单轨双天线横向模式,图片引自
        \inlinecite{niuSARInSARJiZhuYongYuKuangQuTanCeYuXingBianJianCeYanJiu2015}。}
        \label{fig:sar1}
    \end{figure}
    \item 单轨双天线纵向模式。
    单轨双天线纵向模式和轨双天线横向模式的区别仅仅在于
    两副天线和飞行器飞行轨道平行。如图\ref{fig:sar2}所示。
    这种测量模式一般在机载的SAR传感器上实现。
    用于监测地面物体的运动。
    \begin{figure}[htb]
        \centering
        \includegraphics[width=0.8\textwidth]{sar2.png}
        \caption{单轨双天线纵向模式,图片引自
        \inlinecite{niuSARInSARJiZhuYongYuKuangQuTanCeYuXingBianJianCeYanJiu2015}。}
        \label{fig:sar2}
    \end{figure}
    \item 单天线重复轨道模式。
    这种工作模式的SAR传感器一般只搭载在卫星上。
    飞行器上只搭载一副天线,多次对同一地点观测,
    所以需要较为精确的飞行器轨道信息。如图\ref{fig:sar3}所示。
    目前星载SAR干涉成像主要采用这种工作模式。
    \begin{figure}[htb]
        \centering
        \includegraphics[width=0.8\textwidth]{sar3.png}
        \caption{单天线重复轨道模式,图片引自
        \inlinecite{niuSARInSARJiZhuYongYuKuangQuTanCeYuXingBianJianCeYanJiu2015}。}
        \label{fig:sar3}
    \end{figure}
\end{enumerate}
目前常用的SAR卫星见表\ref{tab:sarsatellite}。
\begin{table}
    \centering\small
    \begin{tabular}{@{}cccc@{}}
    \toprule
    卫星                                 & 发射时间      & 波段 & 国家  \\ \midrule
    SEASAT & 1978 & L & 美国 \\
    ERS-1 &1991 & C &欧盟 \\
    JERS-1 & 1992 & L & 日本 \\
    ERS-2 & 1995 & C & 欧盟 \\
    RADARSAT-1 & 1995 & C & 加拿大 \\
    ENVISAT & 2002 & C & 欧洲 \\
    ALOS-1 & 2006 & L & 日本 \\
    TerreaSAR-X/TanDEM-X               & 2007      & X  & 德国  \\
    RADARSAT-2                         & 2007      & C  & 加拿大 \\
    COSMO-SkyMed 1-4                   & 2007-2010 & X  & 意大利 \\
    RISAT-1                            & 2012      & C  & 印度  \\
    环境一号C卫星(HJ-1C)                     & 2012      & S  & 中国  \\
    Kompsat-5                          & 2013      & X  & 韩国  \\
    PAZ                                & 2013      & X  & 西班牙 \\
    Sentinel-1                         & 2013      & C  & 欧盟  \\
    ALOS2                              & 2013      & L  & 日本  \\
    高分三号                               & 2016      & C  & 中国  \\
    RADARSAT Constellation Mission 1-3 & 2017/2018 & C  & 加拿大 \\ \bottomrule
    \end{tabular}
    \caption{常用的SAR卫星}
    \label{tab:sarsatellite}
\end{table}

\section{InSAR简介}
每幅SAR影像的数据不仅包含地面散射光波的强度信息,也包含了与传播距离的相位信息。
即
\begin{equation}
    \phi=-\frac{4\pi}{\lambda}r
\end{equation}
如果对于同一块区域有两幅影像,就可以对两幅影像做干涉处理。
原理如图\ref{fig:geometry}所示。
\begin{figure}[htb]
    \centering
    \includegraphics[width=0.5\textwidth]{geometry.png}
    \caption{InSAR技术原理示意图,图片引自
    \inlinecite{niuSARInSARJiZhuYongYuKuangQuTanCeYuXingBianJianCeYanJiu2015}。}
    \label{fig:geometry}
\end{figure}
所谓干涉,就是指二者相位相减。即
\begin{equation}
    \Delta \phi=-\frac{4\pi}{\lambda}(r_1-r_2)
\end{equation}
其中$r_1-r_2$为卫星视线方向的距离变化。
注意这里讨论的只是忽略大气效应即其他失相干效应,只考虑理想情况。
通常情况下,两幅影像并不是在相同的位置拍摄,二者之间有一定的距离。
所以当不存在任何的地表形变的情况下,可以通过一些几何的参数将此变化表达出来。
\begin{equation}
    \Delta \phi=\frac{4\pi}{\lambda}\left(\sqrt{r_1^2−2(B_hsinθ_1−B_{\perp}sinθ_1)r_1+B^2}−r_1\right)
\end{equation}
如果已知数字高程模型的话,可以算出来在理想条件下,无形变时的相位差。
将实际算得的相位差和无形变时理论计算的相位差相减,剩余的相位差即为因形变引起的相位差。
\begin{equation}
    \Delta \phi_{r}=-\frac{4\pi}{\lambda}u_{LOS}
\end{equation}
其中,$u_{LOS}$为地表形变在卫星视线(line-of-site:LOS)方向的投影,
从而可以把地表在卫星视线方向的形变算出来。
这种技术称之为D-InSAR。

注意,实际得到的相位是“缠绕”的相位,即相位是处于$(-\pi,\pi]$之间的。
将缠绕的相位通过解缠操作之后才得到上文中使用的相位。
解缠的方法很多,基本的假设是相邻像素的相位差不超过$\pi$。

诸多实例(火山喷发变形,地震震前和震后变形,地面沉降)
均证明InSAR是一种有效的测量地表形变的技术。

\section{PS-InSAR简介}
两幅SAR影像间隔时间可能很长,如果地表有植被,或者一年中有部分时间被冰雪覆盖,
地表散射体在这段时间内有较大的变化,有可能导致干涉图的失相干。
如果能挑出地表散射体中较稳定的部分,如建筑物,道路桥梁,岩石等
永久散射体(persist scatterer)反射波的相位差,剔除掉不稳定的散射体反射波的相位差,
经相位解缠之后就能得到较可靠形变结果。
这正是PS-InSAR技术的初衷。

2001年,Ferretti等人首次提出PS-InSAR\cite{ferrettiNonlinearSubsidenceRate2000}技术。
即在一组时间序列SAR影像中选取相干性和稳定性均较好的永久散射体。
通过这些PS点的相位信息获得地表形变信息。
2004年,Andy hopper提出了新的PS算法\cite{hooperNewMethodMeasuring2004},
并将其实现在StamPS软件中\cite{hooperRecentAdvancesSAR2012}。
本研究使用这一软件获得形变信息。

PS-InSAR技术的主要内容为PS点的选取,分为两个步骤:
\begin{enumerate}
    \item 初步筛选。
    这一步通过分析模长的稳定性做筛选。
    Ferretti等人证明了,对于反射波的复信号,当相位的标准差较小时,有
    \begin{equation}
        \sigma_\varphi \approx \frac{\sigma_a}{\mu_a}
    \end{equation}
    其中,$\sigma_\varphi$是相位标准差,$\sigma_a$是反射波振幅的标准差,$\mu_a$是振幅的均值。
    通过设置$\frac{\sigma_a}{\mu_a}$的上界$D_A$,筛选出一批PS点。
    \item 再次筛选。
    这一步通过分析相位的稳定性来做筛选。
    首先通过相位滤波得到噪声相位,对噪声相位做进一步的筛选选出最终的PS点。
\end{enumerate}

PS-InSAR的输入数据为连续的$M$幅SAR影像以及数字高程模型(DEM)。
首先,选取某一幅影像为主影像,然后通过配准,差分干涉,
以及PS点的选取等操作得到$M-1$幅干涉图,
之后通过相位解缠得到形变图以及平均形变速率图。

PS点的差分干涉相位中主要的误差来源于高程误差,大气影响等因素,每个PS点的相位组成可以表达成:
\begin{equation}
    \varphi_{diff}=\varphi_{topo}+\varphi_{def}+\varphi_{orb}+\varphi_{atm}+\varphi_{noise}
\end{equation}
其中
\begin{itemize}
    \item $\varphi_{topo}$为由于所采用的数字高程模型DEM不够准确而造成的误差,可以证明
    \begin{equation}
        \varphi_{topo}=-\frac{4\pi B_{\bot}}{\lambda R sin \theta}\epsilon
    \end{equation}
    \item $\varphi_{def}$为地表形变导致的差分干涉相位的变化,一般来说是研究所需要的结果,其余的各项均为误差;
    \item $\varphi_{orb}$是卫星轨道定轨不精确,轨道参数不准确引入的误差;
    \item $\varphi_{atm}$为大气的延迟效应导致的误差;
    \item $\varphi_{noise}$为其他的一些噪声效应引入的误差,如热噪声等。
\end{itemize}

\section{SBAS-InSAR简介}
SBAS是另外一种利用多个SAR影像数据的时间序列InSAR方法。
SBAS通过设置选取SAR影像配对的时空基线的上限,
将所有的SAR影像分为了不同组,每组的基线较小,从而避免了空间上的失相干性,减小了失相干和高程模型误差的影响。

这里有必要简要叙述一下SBAS处理数据的算法。
假设一共有$n+1$景SAR影像数据,获取的时间分别为$t_0,t_1,\ldots,t_n$。
选取某一景影像为超级主影像。
满足时空基线限制的影响对共有$M$对。分别为$(A_1,B_1),(A_2,B_2),\ldots,(A_M,B_M)$。
则
\begin{equation}
    \Delta\phi_j=\phi_{A_j}-\phi_{A_j}
\end{equation}
其中,$j=1,2,\ldots,M$,$\phi_j$为第$j$景影像相对于超级主影像的差分相位,
$\Delta\phi_j$为第$j$组影像对的差分干涉相位,共计$M$个方程。
由已知的$\phi$根据以上方程求得$\Delta\phi$,
然后由得到的$\Delta\phi$反过来求$\phi$。
由于共有$M$个方程,$n$个未知数,方程组有无穷多组解,使用最小二乘法求最小二范数的解。

\section{StamPS软件简介}
StamPS是基于Matlab的一套软件,软件处理数据一共分为8步:
\begin{enumerate}
    \item 读取数据,并保存在matlab工作区;
    \item 估计相位噪声,这一步通过循环估计每一幅干涉图的每一个像素点的相位噪声;
    \item 通过分析各像素点的噪声情况筛选出PS点,这一步还估计出非PS点的像素的密度;
    \item 上一步只是粗略地选择PS点,其中还有较多的不稳定的像素点,这一步清除这些点;
    \item 去除所选择的像素点相位中空间不相关的视角误差;
    \item 相位解缠;
    \item 估计空间相关的视角误差,该误差绝大多数来源于空间相关的DEM误差,
    包括DEM本身的误差以及从地面坐标到雷达坐标投影的误差;
    \item 去除大气效应误差。
\end{enumerate}