% !TeX root = ../main.tex

\ustcsetup{
  keywords = {
    SAR, InSAR, Mogi, 永久散射体干涉(PS-InSAR), 卤水井, 下沉, 建模
  },
  keywords* = {
    SAR, InSAR, Mogi, Persist Scatter Interferometry(PS-InSAR), 
    Brine well, Subsidence, Modeling
  },
}

\begin{abstract}
  大尺度的地表沉降是一种严重的灾害。
  随着城市的建设,地面沉降以及其所造成的次生灾害
  对城市的发展建设的危害越来越大,
  更严重的是,这种灾害没有明显的前兆并且很难提前做出预测。
  
  本课题拟通过新墨西哥州某一沉降卤水矿的InSAR反演的形变结果,做出沉降的物理模型。
  比较InSAR反演的模型和同一研究区域其他研究的成果,分析InSAR在解决此类问题上的优劣性,
  并对未来的InSAR技术在相关领域的作用做了展望。

  本文对新墨西哥州卤水井沉降的研究,能够为我国类似的灾害的检测和治理提供参考。
\end{abstract}

\begin{abstract*}
  Large scale land subsidence is a serious disaster. 
  With the construction of the city, 
  the land subsidence and the secondary disasters caused by it 
  are more and more harmful to the development and construction of the city. 
  More seriously, this kind of disaster has no obvious precursor 
  and is difficult to predict in advance.

  In this paper, a physical model of subsidence is proposed 
  based on the InSAR deformation results of a collapsed brine mine in New Mexico. 
  By comparing the model based on InSAR result with other research results in the same research area, 
  the advantages and disadvantages of InSAR in solving such problems are analyzed, 
  and the future role of InSAR technology in related fields is prospected.

  the study of brine well subsidence in New Mexico in this paper 
  provide reference for the detection and treatment of similar disasters in China.
\end{abstract*}
