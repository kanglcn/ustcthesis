\chapter{总结与展望}
本文首先阐述了InSAR技术的相关原理,然后利用ALOS PALSAR的数据
将SBAS-InSAR技术应用于新墨西哥州卤水井沉降的监测与建模上,
揭示了该卤水井沉降的情况。
验证了SBAS-InSAR方法在解决此类问题上的有效性。
之后利用沉降的数据对当地地下区域做了理论建模,得到了比较准确的模型。
能够对此形变给出比较有说服力的解释。
与其他研究成果相比,更突出了本研究使用的方法的准确性。

本研究的不足之处主要有以下几点:
\begin{enumerate}
    \item 研究区域较小,数据量不足,反演的结果准确性不高。
    \item 本研究只做了最简单的点源模型,和实际情况出入较大,采用和实际空洞更贴近的模型效果应该会更好。
\end{enumerate}

传统的地表形变监测方式成本较高,且数据量少,监测的精度也达不到要求。
InSAR作为一种的监测技术,成本低且精度足够高,是未来应当大力推广的监测方式。

中国是个资源大国,自然资源的开采很有可能带来一些次生灾害。
大规模的城市建设也同样有可能导致地面的沉降。
事实证明,InSAR是监测有地表形变征兆的灾害的很好的方法,
可以为地质灾害评估以及制定对应的政策提供数据支持。