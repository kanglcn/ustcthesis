\chapter{总结与展望}
随着经济与社会的不断发展,自然资源被加速开采与利用,
不可避免地会引发一定的次生灾害。
随着城市的发展,此类灾害对生产和生活的影响越来越大。
地面塌陷是一种危害严重的自然灾害,并且前兆不明显,一旦坍塌速度极快,
会造成相当严重的人力物力损失。
地面沉降是某些地面塌陷的前兆,具有分布范围广,持续时间长的特点。
如何预测,预防此类灾害是当今社会迫切需要解决的问题。
传统的沉降监测手段成本高,监测精度不高,所得数据较少,
不能满足大尺度,长时间的监测要求。

合成孔径雷达干涉技术 (InSAR) 技术,具有大尺度观测的特点,
且不需要任何地面上的人工操作,能对同一区域进行重复的卫星观测,
而且测得的形变较为精密,能很好地满足地面沉降监测的要求,从而对突发灾害作出预报。
我国某些区域尤其是矿产资源比较丰富的地区,面临着坍塌的风险。
本研究所使用的方法,可以对此类灾害起到一定的预警作用。

本文首先阐述了InSAR技术的相关原理,然后利用ALOS PALSAR的数据
将SBAS-InSAR技术应用于新墨西哥州卤水井沉降的监测与建模上,
揭示了该卤水井沉降的情况。
验证了SBAS-InSAR方法在解决此类问题上的有效性。
之后利用沉降的数据对当地地下区域做了理论建模,得到了比较准确的模型。
能够对此形变给出比较有说服力的解释。
与其他研究成果相比,更突出了本研究使用的方法的准确性。

本研究的不足之处主要有以下几点:
\begin{enumerate}
    \item 研究区域较小,数据量不足,反演的结果准确性不高;
    \item 本研究只做了最简单的点源模型,和实际情况出入较大,采用和实际空洞更贴近的模型效果应该会更好。
\end{enumerate}

传统的地表形变监测方式成本较高,且数据量少,监测的精度也达不到要求。
InSAR作为一种的监测技术,成本低且精度足够高,是未来应当大力推广的监测方式。

中国是个资源大国,自然资源的开采很有可能带来一些次生灾害。
大规模的城市建设也同样有可能导致地面的沉降。
本研究和一系列他人的研究成果表明,InSAR是监测有地表形变征兆的灾害的很好的方法,
可以为相关地质灾害评估,灾害起因研究以及制定对应的政策提供数据支持。